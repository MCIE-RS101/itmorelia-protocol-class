%!TeX root=./thesisStructure.tex
%Add here the packages, variables, definitions, environments and specific elements created only for your thesis

% 1 Code listing setup
%------------------
\RequirePackage{listings}
\newcaptionname{spanish}{\lstlistlistingname}{Sección de Códigos}
\renewcommand{\lstlistingname}{Código}
\renewcommand{\lstlistlistingname}{Códigos}
\lstset{ frame=Ltb,
     framerule=0pt,
     aboveskip=0.2cm,
     framextopmargin=3pt,
     framexbottommargin=3pt,
     framexleftmargin=0.4cm,
     framesep=0pt,
     rulesep=.4pt,
     backgroundcolor=\color{gray97},
     rulesepcolor=\color{black},
     %
     stringstyle=\ttfamily,
     showstringspaces = false,
     %basicstyle=\small\ttfamily,
     %basicstyle=\small,
     commentstyle=\color{gray45}\ttfamily,
     keywordstyle=\bfseries,
     %
     numbers=left,
     numbersep=15pt,
     numberstyle=\tiny,
     numberfirstline = false,
     breaklines=true,
   }

% 2 Commands and variables
% 2.1 Thesis's title page config
%------------------
\newcommand{\myTitle}{Un Título de la Tesis no Muy Complicado y Largo Pero Capaz de Definir Concretamente}
\newcommand{\mySubtitle}{Caso de Estudio a Nivel Maestría}
\newcommand{\myDegree}{Maestría en Ciencias en Ingeniería Electrónica}
% 2.2 Autores:
\newcommand{\myName}{Gerardo Marx Chávez-Campos}
\newcommand{\myNumber}{D01120294}
%\newcommand{\myPartner}{Otro autor}
%----------------
% 2.3 Mesa de revisión
%Director:
\newcommand{\myProf}{Homer Simpson}
%Codirector:
\newcommand{\myOtherProf}{James Clerk Maxwell \xspace}
%Revisor 1:
\newcommand{\mySupervisor}{Piere Simon Laplace\xspace}
%Revisor 2:
\newcommand{\myOtherSupervisor}{Jean-Baptiste Joseph Fourier\xspace}
%--------------
% 2.4 Datos del instituto:
\newcommand{\myFaculty}{Maestría en Ingeniería Electrónica}
\newcommand{\myDepartment}{División de Estudios de Posgrado e Investigación}
\newcommand{\myUni}{Instituto Tecnológico de Morelia}
\newcommand{\myLocation}{Morelia, Michoacán, México}
\newcommand{\myTime}{Abril 2021}
\newcommand{\myThesisVersion}{Rev 2.0}

%---------------------
% 3	USEFUL COMMANDS
%---------------------
\newcommand{\ie}{i.\,e.}
\newcommand{\Ie}{I.\,e.}
\newcommand{\eg}{e.\,g.}
\newcommand{\Eg}{E.\,g.}


%-------------------------
% 4. Required packages for
% figures, plots, subfigures
%-------------------------
\usepackage{float}
\usepackage{subfigure}
\usepackage{tikz}
\usepackage{booktabs} %Beautiful tables
\usepackage{todonotes} %Todo list for comments
\usepackage[version-1-compatibility]{siunitx} %easy writting of SI units
\usepackage{pgfplots}
\pgfplotsset{compat=1.10}
\decimalpoint %Decimal point enable
\usepackage[acronym]{glossaries}
%-------------------------
% 5. setting space for indentation
% space between paragraphs and
% baseline.
% Uncomment to set parameters
%-------------------------
% \setlength{\parindent}{4em}
\setlength{\parskip}{1em}
%\renewcommand{\baselinestretch}{2.0} % Interlineado
