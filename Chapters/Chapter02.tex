%!TeX root=../thesisStructure.tex
% Chapter 2
\chapter{Ejemplos de Entornos} % Chapter title
\label{ch:MarcoTeorico} 

\section{Figuras y gráficas}
Agregar la cantidad de capítulos que se requieran. En la \autoref{fig:A} se muestra la letra A.


\blindtext[1] %Texto Dummy


\begin{figure}[!h]
	\centering
	\includegraphics[width=2.5in]{example-image-a}
	\caption{Ejemplo de como insertar figuras sencillas.}
	\label{fig:A}
\end{figure}

\blindtext[1] %Texto Dummy




\begin{figure}[!h]%
\centering
\subfigure[Figure A]{%
\label{fig:fig-a}%
\includegraphics[height=2in]{example-image-a}}%
\\%New line for figure
\subfigure[Figure B]{%
\label{fig:fig-b}%
\includegraphics[height=1.5in]{example-image-b}}%
~ %Space between figure
\subfigure[Figure C]{%
\label{fig:fig-c}%
\includegraphics[height=1.5in]{example-image-c}}%
\caption{Using \texttt{subfigures} package.}
\label{fig:subFigures}
\end{figure}


Las cantidades usan comas y punto decimal correctamente para México. Es decir 1,526.00, aparece correctamente tanto como texto, como ecuación $1,526.00$, cuando se agrega el comando \texttt{decimalpoint} en la configuración; test.

\section{Agregando Acrónimos y Glosario}
A continuación se muestran algunos ejemplos del uso del glosario y los comandos que lo invocan. Por ejemplo, para hablar del lenguaje \Gls{latex} y su especial aplicación en todo tipo de documentos que contienen ecuaciones \gls{maths}. Las \Glspl{formula} que aparecen en los documentos son renderizadas de forma adecuada y fácil una vez que se acostumbra al uso de los comandos.

	Por otro lado, dado un conjunto de resultados numéricos, existen diferentes métodos básicos para calcular el \acrlong{ecm}, el cual se abrevia como \acrshort{ecm}. Este error es común de usarse en en el cálculo de las estimaciones por \acrfull{ls}.