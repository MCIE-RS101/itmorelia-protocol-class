%!TeX root=../thesisStructure.tex
% Chapter 2
\chapter{Ejemplos de Entornos} % Chapter title
\label{ch:MarcoTeorico} 

\section{Figuras y gráficas}
Agregar la cantidad de capítulos que se requieran. En la \autoref{fig:A} se muestra la letra A.


\blindtext[1] %Texto Dummy


\begin{figure}[!h]
	\centering
	\includegraphics[width=2.5in]{example-image-a}
	\caption{Ejemplo de como insertar figuras sencillas.}
	\label{fig:A}
\end{figure}

\blindtext[1] %Texto Dummy




\begin{figure}[!h]%
\centering
\subfigure[Figure A]{%
\label{fig:fig-a}%
\includegraphics[height=2in]{example-image-a}}%
\\%New line for figure
\subfigure[Figure B]{%
\label{fig:fig-b}%
\includegraphics[height=1.5in]{example-image-b}}%
~ %Space between figure
\subfigure[Figure C]{%
\label{fig:fig-c}%
\includegraphics[height=1.5in]{example-image-c}}%
\caption{Using \texttt{subfigures} package.}
\label{fig:subFigures}
\end{figure}


Las cantidades usan comas y punto decimal correctamente para México. Es decir 1,526.00, aparece correctamente tanto como texto, como ecuación $1,526.00$, cuando se agrega el comando \texttt{decimalpoint} en la configuración; test.

\section{Agregando Acrónimos y Glosario}
A continuación se muestran algunos ejemplos del uso del glosario y los comandos que lo invocan. Por ejemplo, para hablar del lenguaje \Gls{latex} y su especial aplicación en todo tipo de documentos que contienen ecuaciones \gls{maths}. Las \Glspl{formula} que aparecen en los documentos son renderizadas de forma adecuada y fácil una vez que se acostumbra al uso de los comandos.

Por otro lado, dado un conjunto de resultados numéricos, existen diferentes métodos básicos para calcular el \acrlong{ecm}, el cual se abrevia como \acrshort{ecm}. Este error es común de usarse en en el cálculo de las estimaciones por \acrfull{ls}.

\section{Secciones de código}


\begin{lstlisting}
import numpy as np
    
def incmatrix(genl1,genl2):
    m = len(genl1)
    n = len(genl2)
    M = None #to become the incidence matrix
    VT = np.zeros((n*m,1), int)  #dummy variable
    
    #compute the bitwise xor matrix
    M1 = bitxormatrix(genl1)
    M2 = np.triu(bitxormatrix(genl2),1) 

    for i in range(m-1):
        for j in range(i+1, m):
            [r,c] = np.where(M2 == M1[i,j])
            for k in range(len(r)):
                VT[(i)*n + r[k]] = 1;
                VT[(i)*n + c[k]] = 1;
                VT[(j)*n + r[k]] = 1;
                VT[(j)*n + c[k]] = 1;
                
                if M is None:
                    M = np.copy(VT)
                else:
                    M = np.concatenate((M, VT), 1)
                
                VT = np.zeros((n*m,1), int)
    
    return M
\end{lstlisting}

%%% Local Variables:
%%% mode: latex
%%% TeX-master: "../thesisStructure"
%%% End:
